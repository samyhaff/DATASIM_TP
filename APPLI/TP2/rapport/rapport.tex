\documentclass[12pt,a4paper,titlepage]{scrartcl}

\usepackage{preamble}

\title{TP Traitement du signal et problèmes inverses}
\subtitle{Applications sur des exemples EDF}
\author{Saâd Aziz Alaoui, Yassine Jamoud, Samy Haffoudhi}
\date{\today}

\begin{document}

\maketitle

\section*{Introduction}

Lors de ce TP nous allons explorer trois problèmes de traitement
du signal en lien avec des exemples EDF. Le premier problème portera
sur le traitement de signaux multicapteurs illustré sur des données
de température issues de mesures de fibres optiques. Ensuite, nous
nous intéresserons au problème inverse de surrésolution à partir de
données de fibre optique déformation. Enfin, nous verrons le
problème inverse d'estimation des sources à partir de données
de contrôle non destructif ultrasonores.

Pour chacun de ces exemples,
nous commencerons tout d'abord par une présentation du contexte et
de l'enjeu avant de jouer sur les paramètres des différentes méthodes
pour bien appréhender leurs effets.

% \begin{figure}[H]
%     \caption{TITRE}
%     \includegraphics[width=\textwidth]{nom_fichier}
%     \centering
% \end{figure}

\section{Problème inverse de sources à partir de données de contrôle
non-destructif ultrasonore}

\subsection{Le contexte d'application}

L'objectif de cette partie est de réaliser un contrôle non-destructif
de défauts dans des soudures. Pour ce faire, on utilise des signaux
ultrasonores induits par la soudure à inspecter.

On souhaite, à l'aide de ces signaux, détecter des perturbations dues
à ces défauts. Pour ce faire, nous disposons d'une ondelette de
référence. Elle est réalisée par calibration à l'aide de 4 trous de
1 millimètre.

On peut donc détecter des défauts similaires à des trous d'un millimètre.
En effet, par déconvolution, on peut retrouver les positions des défauts
(à un décalage près du à la convolution)

% figure 01

Pour réaliser la déconvolution,on utilise une minimisation de l'erreur
quadratique moyenne. En revanche, sans information à priori sur la
réflectivité, on obtient des résultats de mauvaise qualité. On utilise
alors une régularisation.

% formule latex

On obtient donc une nette amélioration mais le défaut apparait avec un
double pic

% figure 02

Pour tenir compte de la possible déformation en phase de l'ondelette,
on a recours à la transformée de Hilbert de l'ondelette $h$. On obtient
alors un nouveau modèle de convolution ainsi qu'un nouveau critère
à minimiser.

% formule latex

% figure 03

On peut finalement observer le module des deux solutions sur le graphique
ci-dessous :

% figure 04

\subsection{Influence des paramètres}

On constate sur la figure ci-dessus que le choix des paramètres n'est
clairement pas optimal.

En effet, pour ce problème nous disposons de deux paramètres, $\lambda$
et $\mu$ :

\begin{itemize}
    \item{Le paramètre $\mu$ permet de s'assurer de l'existence de la solution et
        doit être non nul. Plus il est proche de 0, plus la solution
        est optimale en revanche, l'optimisation sera plus complexe. On
        choisit alors une valeur $\mu = 10^{-2}$ pour obtenir un bon
        compromis.}
    \item{Le paramètre $\lambda$ correspond au paramètre de régularisation.
        Il contrôle le poids du terme de régularisation.}
\end{itemize}

Dans la partie précedente, nous avons choisi une valeur très faible pour
le paramètre de régularisation : $\lambda = 10^{-4}$. On obtenait donc
uen solution non optimale. En effet pour des valeurs faibles de $\lambda$,
la régularisation aura un poids faible tandis qu'au contraire,
pour des valeurs trop importantes de $\lambda$, on surrégularise.

On constante alors que pour une valeur $\lambda = 10^{-3}$, on obtient
les meilleurs résultats.

Pour la solution réelle :

% figure 2 opti

Et pour la solution complexe :

% figure 3 opti

Et enfin, pour les modules :

% figure 4 opti

Pour des valeurs de $\lambda$, plus élevées, les pics sont trop épais
et on perd donc en résolution.

Remarque : On peut observer sur la FIGURE 4 OPTI, la solution complexe
est bien plus intéressante que la solution réelle. En effet, on obtient
bien un seul pic par défaut et on remarque aussi que pour le premier
défaut, la solution complexe nous donne 2 pics. Il s'agit d'un défaut
différent qui ne correspond pas à un trou (un défaut de surface) qui
nécessite alors une autre ondelette afin d'être correctement detecté.

\end{document}
