\documentclass[12pt,a4paper,titlepage]{article}

\usepackage{preamble}

\title{Fetal ECG extraction}
\author{Yassine Jamoud, Samy Haffoudhi}
\date{\today}

\begin{document}

\maketitle

\section*{Introduction}

Le but de ce TP est d'extraire un ECG fœtal à partir de mesures issues de capteurs abdominaux.
Nous allons commencer par l'analyse et la détection de pics R d'un ECG sain. Puis nous 
passerons à l'extraction d'un ECG fœtal en implémentant deux méthodes complémentaires : 
le filtrage spatial par PCA (periodic component analysis ) et le filtre de Kalman étendu.

\section{Healthy ECG signal analysis}

\subsection{Basic analysis}

Commençons par représenter le signal ECG considéré:

\begin{figure}[H]
    \caption{Signal ECG}
    \includegraphics[width=\textwidth]{fig1}
    \centering
\end{figure}

On reconnait bien l'allure générale d'un ECG sain. On peut en particulier, distinguer les 
différentes ondes et observer la quasi-périodicité du spectre.

Représentons le spectre associé :

\begin{figure}[H]
    \caption{Spectre de l'ECG}
    \includegraphics[width=\textwidth]{fig2}
    \centering
\end{figure}

On remarque que les composantes fréquentielles du signal correspondent à de basses
fréquences, on se restreint à un tracé du spectre entre 0 et 50 Hz. Par ailleurs, les
amplitudes sont globalement décroissantes avec la fréquence et le spectre s'apparente 
à un peigne de Dirac, ce qui s'explique par la quasi-périodicité du signal.

Le spectre ne permettant pas d'obtenir des informations temporelles, on s'intéresse 
maintenant au spectrogramme du signal.

Représentons le spectrogramme pour trois choix différents de taille de fenêtre :

\begin{figure}[H]
    \caption{Spectrogramme de l'ECG}
    \includegraphics[width=\textwidth]{fig3}
    \centering
\end{figure}

On peut observer l'influence de la taille de la fenêtre sur l'allure du spectrogramme obtenu.
En effet, pour une fenêtre plus courte, d'une durée de 150ms, on observe un étalement en fréquences tandis que
pour une durée plus importante, 2s, on a cette fois un étalement moins important en fréquences.
Cette dualité temps-fréquence nécessite alors un choix de paramètre adapté à l'information
recherchée sur le spectrogramme.
Par ailleurs, pour la fenêtre de durée 15s on peut observer clairement un spectre des raies mais de
fréquence variant légèrement en fonction du temps car le signal n'est pas parfaitement
périodique/

\subsection{R peak detection}

\subsubsection{Design and analysis of filters}

Le processus de détection des pics R repose sur l'utilisation de plusieurs filtres.

\begin{itemize}
    \item{Un filtre passe-bas : $H(z) = \frac{(1-z^{-n})^2}{(1-z^{-1})^2}$ avec
        $n = \frac{F_s}{F_c} = \frac{1000}{40} = 25$, n impair pour avoir un retard entier}
    \item{Un filtre passe-haut : $H(z) = nz^{-\frac{n-1}{2}} - \frac{1-z^{-1}}{1-z^{-1}}$
        avec $n = \frac{F_s}{F_c} =\frac{1000}{8} = 125$, n impair pour avoir un retard entier.}
    \item{Un filtre dérivateur}
    \item{Un filtre MA : $H(z) = \frac{1}{n} \frac{1-z^{-n}}{1-z^{-1}}$ avec 
            $n = 149$. En effet, pour $F_s$ = 200 Hz, les auteurs de l'article ont $n = 30$,
        on choisit 149 pour obtenir n impair.}
\end{itemize}

Représentons les réponses des différents filtres :

\begin{figure}
\begin{subfigure}{.5\textwidth}
  \centering
  \includegraphics[width=.8\linewidth]{fig4}  
  \caption{passe-bas}
  \label{fig:sub-first}
\end{subfigure}
\begin{subfigure}{.5\textwidth}
  \centering
  \includegraphics[width=.8\linewidth]{fig5}  
  \caption{passe-haut}
  \label{fig:sub-second}
\end{subfigure}

\newline

\begin{subfigure}{.5\textwidth}
  \centering
  \includegraphics[width=.8\linewidth]{fig6}  
  \caption{MA}
  \label{fig:sub-third}
\end{subfigure}
\begin{subfigure}{.5\textwidth}
  \centering
  \includegraphics[width=.8\linewidth]{fig7}  
  \caption{Dérivateur}
  \label{fig:sub-fourth}
\end{subfigure}
\caption{Réponses fréquentielles}
\label{fig:fig}
\end{figure}

On observe alors que les tracés pour les différents filtres ont bien les allures attendues, 
avec les bonnes fréquences de coupure et une phase linéaire pour le passe-bas et le 
passe-haut.

Le terme $z^{-2}$ pour le filtre déviateur est nécessaire car il permet d'assurer la
causalité du filtre, ce qui assure que le système ne dépende pas du futur.

\subsubsection{Algorithm to detect R peaks}

Représentons les résultats obtenus par les algorithmes de détection de pics R :

\begin{figure}[H]
    \caption{Résultats des algorithmes}
    \includegraphics[width=\textwidth]{fig8}
    \centering
\end{figure}

On observe alors sur la figure ci-dessus que :

\begin{itemize}
    \item{Chacun des deux algorithmes pris séparément détecte les pics R mais aussi les
            pics T}
    \item{La prise en compte des deux algorithmes permet d'éliminer le détection des pics
            T}
    \item{La détection n'est quand même pas parfaite. On voit que les pics sont parfois
            détectés avec un léger décalage}
\end{itemize}

Ainsi, ces algorithmes simples à mettre en œuvre et peu couteux en calculs permettent 
d'obtenir des résultats satisfaisants sur un ECG sain.

\section{Fetal ECG extraction}

\subsection{Spatial filtering: $\pi$CA}

On commence par filtre les données afin de retirer la ligne de base. On choisit une
fréquence de coupure $f_c$ = 40 Hz et on a alors $N = \frac{1000}{40} = 25$.

En implémentant les différnts algorithmes, on obtient les résultats suivants :

\begin{figure}[H]
    \caption{Résultats des algorithmes}
    \includegraphics[width=\textwidth]{fig9}
    \centering
\end{figure}

On observe alors que :

\begin{itemize}
    \item{On a pu séparer les deux ECG dans la majorité des cas.}
    \item{Pour le premier canal, où les données semblent le plus bruité, l'algorithme
            PiCA n'a pas permis de séparer les deux ECG.}
    \item{Pour le canal 4, c'est l'algorithme PCA qui fournit cette fois le moins bon
            résultat en amplifiant le bruit.}
    \item{L'algorithme ICA est le moins performant sur le canal 2 puisqu'il 
            amplifie le bruit.}
\end{itemize}

Ainsi, aucun des algorithmes n'apparait clairement comme plus efficace que les autres,
les résultats dépendent des données en entrée.

Restregnons nous maintenant aux deux premiers canaux et affichant les résultats :

\begin{figure}[H]
    \caption{Résultats des algorithmes}
    \includegraphics[width=\textwidth]{fig10}
    \centering
\end{figure}

On voit que les résultats pour les algorithmes PCA et ICA sont moins bons que lorsqu'on
disposait des données des 4 capteurs. Par exemple pour le canal 2, le bruit est amplifié
par l'algorithme PCA, ce qui n'était pas le cas précédemment.

\subsection{ECG denoising by extended Kalman filtering}

Le principe de cette dernière méthode est à partir de données bruitées et d'un modèle
d'observation, non forcement linéaire, obtenir de meilleures données.

Appliquons maintenant la méthode du filtre de Kalman étendu.
On choisit de considèrer les données du 4ème canal contenant l'ECG du fœtus. 
On obtient les résultats suivants :

\begin{figure}[H]
    \caption{Résultats du filtre de Kalman étendu}
    \includegraphics[width=\textwidth]{fig11}
    \centering
\end{figure}

On observe que pour les 4 canaux, c'est l'ECG de la mère qui est extrait et non pas
celui du fœtus. Ce qui peut laisser penser à une erreur d'implémentation de notre part. 

\section*{Conclusion}

Ainsi, lors de TP nous avons pu mettre en œuvre différents algorithmes permettant l'analyse
de signaux ECG. Tout d'abord lors de la première partie, nous avons pu observer différentes
propriétés des signaux ECG à l'aide du tracé spectre et du spectrogramme, telles que la quasi-périodicité. 
Nous avons alors par la suite, pu implémenter
un processus permettant la détection de pics R à l'aide de l'utilisation de différents
filtres. Ce processus simple permet toutefois d'obtenir de très bons résultats. Ensuite, nous
avons implémenté les algorithmes $\pi$CA, PCA et ICA pour réaliser l'extraction de l'ECG d'un
fœtus. On a alors observé qu'aucun des algorithmes ne semble se démarquer et qu'il convient
de tous les tester sur les données pour choisir celui fournissant le meilleur résultat. Aussi,
l'algorithme $\pi$CA a fourni de meilleurs résultats lors de l'utilisation des 4 canaux 
qu'uniquement 2 de ces canaux. Enfin, la méthode du filtre de Kalman étendu permet
de bien éliminer le bruit et d'obtenir un signal ECG clair en sortie mais dans notre cas,
les résultats obtenus ne sont pas satisfaisants puisqu'ils ne correspondent pas à l'ECG
du fœtus.

\end{document}
