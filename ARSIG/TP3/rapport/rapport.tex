\documentclass[12pt,a4paper,titlepage]{article}

\usepackage{preamble}

\title{Transformée discrête en ondelettes : travaux pratiques}
\author{Yassine Jamoud, Samy Haffoudhi}
\date{\today}

\begin{document}

\maketitle

\section*{Introduction}

Le but de ce TP est de prendre en main la transformée en ondelettes à l'aide de Matlab et de
la boîte à outils \texttt{Wavelab}. Nous allons alors commencer par tracer des fonctions ondelettes,
échelles et une transformée en ondelettes. Nous implémenterons ensuite une procédure de
"débruitage" basée sur la transformée en ondelettes et enfin nous réaliserons de la compression
d'images.

\section{Tracé d'ondelettes et de fonctions échelles par DWT inverse}

\begin{enumerate}

    \item{Sous forme informatique la DWT est représentée sous la forme : 
        $$ \texttt{DWT}(x) = [a_J, d_J, d_{J-1}, \dots, d_1] $$ où $J$ est l'échelle maximale.
    
        Ainsi, $\forall j, k$,

        \begin{itemize}
            \item{Le coefficient $a_J[k]$ est à l'indice k dans la représentation.}
            \item{Le coefficient $d_j[k]$ est à l'indice $ \frac{N}{2^j} + k  $}
        \end{itemize}
    }

\item{ On souhaite construire un vecteur \texttt{DWT x} contenant seulement un coefficient non-nul
        de détail à la plus grande échelle et situé approximativement au milieu de l'axe temporel.

        On place alors d'après 1. ce coefficient non-nul à l'indice :
        $ \frac{N}{2^J} + \frac{N}{2^{J+1}} = \frac{3N}{2^{J+1}} $

    \item{}

    \item{}

    \item{}

    \item{}

    \item{}

\end{enumerate}

\section{Débruitage dans l'espace des ondelettes}

\section{Compression d'images}

\section*{Conclusion}

\end{enumerate}

\end{document}
