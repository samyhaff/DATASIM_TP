\documentclass[12pt,a4paper,titlepage]{article}

\usepackage{preamble}

\title{Transformée discrête en ondelettes : travaux pratiques}
\author{Yassine Jamoud, Samy Haffoudhi}
\date{\today}

\begin{document}

\maketitle

\section*{Introduction}

Le but de ce TP est de prendre en main la transformée en ondelettes à l'aide de Matlab et de
la boîte à outils \texttt{Wavelab}. Nous allons alors commencer par tracer des fonctions ondelettes,
échelles et une transformée en ondelettes. Nous implémenterons ensuite une procédure de
"débruitage" basée sur la transformée en ondelettes et enfin nous réaliserons de la compression
d'images.

\section{Tracé d'ondelettes et de fonctions échelles par DWT inverse}

\begin{enumerate}

    \item{Sous forme informatique la DWT est représentée sous la forme : 
        $$ \texttt{DWT}(x) = [a_J, d_J, d_{J-1}, \dots, d_1] $$ où $J$ est l'échelle maximale.
    
        Ainsi, $\forall j, k$,

        \begin{itemize}
            \item{Le coefficient $a_J[k]$ est à l'indice k dans la représentation.}
            \item{Le coefficient $d_j[k]$ est à l'indice $ \frac{N}{2^j} + k  $}
        \end{itemize}
    }

\item{ On souhaite construire un vecteur \texttt{DWT x} contenant seulement un coefficient non-nul
        de détail à la plus grande échelle et situé approximativement au milieu de l'axe temporel.

        On place alors d'après 1. ce coefficient non-nul à l'indice :
        $ \frac{N}{2^J} + \frac{N}{2^{J+1}} = \frac{3N}{2^{J+1}} $

    \item{}

    \item{}

    \item{}

    \item{}

    \item{}

\end{enumerate}

\section{Débruitage dans l'espace des ondelettes}

\emph{On considèrera dans toute cette partie une ondelette de Haar.}

\begin{enumerate}
    \item{Créons un signal dont la DWT jusqu'à l'échelle  J = 7 contient
            uniquement quelques coefficients de détail non nuls.

            Représentons alors le signal obtenu et sa DWT :

            \begin{figure}[H]
                \caption{Signal et sa DWT}
                \includegraphics[width=\textwidth]{ex2_1}
                \centering
            \end{figure}
        }

    \item{Ajoutons du bruit au signal obtenu et visualisons ce nouveau
            signal ainsi que sa FWT :

            \begin{figure}[H]
                \caption{Signal bruité et sa DWT}
                \includegraphics[width=\textwidth]{ex2_2}
                \centering
            \end{figure}

            On observe ainsi que l'ajout de bruit au signal résulte
            également en un ajout de bruit à sa DWT. Toutefois on
            remarque que les nouveaux coefficients non nuls liés au
            bruit sont d'amplitude faible devant celle des coefficients
            de la DWT du signal sans bruit.Un filtrage des coefficients
            à l'aide d'un seuil devrait permettre un débruitage du signal.
        }

    \item{Mettons maintenant en œuvre le processus de débruitage décrit.
            Représentons alors l'erreur de construction commise en fonction
            de la valeur du seuil $\alpha$ et celle correspondant au seuil
            $\alpha^* = \sigma_b \sqrt{2 ln(N)}$ :

            \begin{figure}[H]
                \caption{Erreur de reconstruction en fonction de alpha}
                \includegraphics[width=\textwidth]{ex2_3}
                \centering
            \end{figure}
        }

    \item{On génère les signaux "Blocks" et "Doppler" à l'aide de la
        fonction \texttt{makesig}.}

    \item{Représentons ces deux signaux ainsi que leur DWT :

            \begin{figure}[H]
                \caption{Singal "Blocks" et sa DWT}
                \includegraphics[width=\textwidth]{ex2_4}
                \centering
            \end{figure}

            \begin{figure}[H]
                \caption{Signal "Dopler" et sa DWT}
                \includegraphics[width=\textwidth]{ex2_5}
                \centering
            \end{figure}

            Ajoutons du bruit avec un RSD de 20 dB et appliquons la
            procédure précédente. On obtient les tracés d'erreur de
            reconstruction en fonction du seuil suivants :

            \begin{figure}[H]
                \caption{Erreur de reconstruction en fonction du bruit (Blocks)}
                \includegraphics[width=\textwidth]{ex2_6}
                \centering
            \end{figure}

            \begin{figure}[H]
                \caption{Erreur de reconstruction en fonction du bruit (Dopler)}
                \includegraphics[width=\textwidth]{ex2_7}
                \centering
            \end{figure}
        }

    \item{En utilisant maintenant une ondelette de Daubechies d'ordre 4,
            on obtient les tracés suivants :

            \begin{figure}[H]
                \caption{Erreur de reconstruction en fonction du bruit (Blocks)}
                \includegraphics[width=\textwidth]{ex2_6bis}
                \centering
            \end{figure}

            \begin{figure}[H]
                \caption{Erreur de reconstruction en fonction du bruit (Dopler)}
                \includegraphics[width=\textwidth]{ex2_7bis}
                \centering
            \end{figure}

        }

\end{enumerate}

\section{Compression d'images}

\section*{Conclusion}

\end{enumerate}

\end{document}
