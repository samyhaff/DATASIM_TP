% This is part of the TFTB Reference Manual.
% Copyright (C) 1996 CNRS (France) and Rice University (US).
% See the file refguide.tex for copying conditions.



\markright{fmsin}
\section*{\hspace*{-1.6cm} fmsin}

\vspace*{-.4cm}
\hspace*{-1.6cm}\rule[0in]{16.5cm}{.02cm}
\vspace*{.2cm}



{\bf \large \sf Purpose}\\
\hspace*{1.5cm}
\begin{minipage}[t]{13.5cm}
Signal with sinusoidal frequency modulation.
\end{minipage}
\vspace*{.5cm}


{\bf \large \sf Synopsis}\\
\hspace*{1.5cm}
\begin{minipage}[t]{13.5cm}
\begin{verbatim}
[y,iflaw] = fmsin(N)
[y,iflaw] = fmsin(N,fmin)
[y,iflaw] = fmsin(N,fmin,fmax)
[y,iflaw] = fmsin(N,fmin,fmax,period)
[y,iflaw] = fmsin(N,fmin,fmax,period,t0)
[y,iflaw] = fmsin(N,fmin,fmax,period,t0,f0)
[y,iflaw] = fmsin(N,fmin,fmax,period,t0,f0,pm1)
\end{verbatim}
\end{minipage}
\vspace*{.5cm}


{\bf \large \sf Description}\\
\hspace*{1.5cm}
\begin{minipage}[t]{13.5cm}
        {\ty fmsin} generates a sinusoidal frequency modulation, whose
minimum frequency value is {\ty fmin} and maximum is {\ty fmax}.  This
sinusoidal modulation is designed such that the instantaneous frequency at
time {\ty t0} is equal to {\ty f0}, and the ambiguity between increasing or
decreasing frequency is solved by {\ty pm1}.\\
 
\hspace*{-.5cm}\begin{tabular*}{14cm}{p{1.5cm} p{8.5cm} c}
Name & Description & Default value\\
\hline
        {\ty N}       & number of points\\
        {\ty fmin}    & smallest normalized frequency          & {\ty 0.05}\\
        {\ty fmax}    & highest normalized frequency           & {\ty 0.45}\\
        {\ty period}  & period of the sinusoidal frequency modulation  & {\ty N}  \\ 
        {\ty t0}      & time reference for the phase           & {\ty N/2} \\
        {\ty f0}      & normalized frequency at time {\ty t0}     & {\ty 0.25}\\
        {\ty pm1}     & frequency direction at {\ty t0} (-1 or +1)& {\ty +1}  \\
  \hline {\ty y}       & signal\\
        {\ty iflaw}   & instantaneous frequency law \\
 
\hline
\end{tabular*}

\end{minipage}
\vspace*{1cm}


{\bf \large \sf Example}
\begin{verbatim}
         z=fmsin(140,0.05,0.45,100,20,0.3,-1.0);
         plot(real(z));
\end{verbatim}
\vspace*{.5cm}


{\bf \large \sf See Also}\\
\hspace*{1.5cm}
\begin{minipage}[t]{13.5cm}
\begin{verbatim}
fmconst, fmlin, fmodany, fmhyp, fmpar, fmpower.
\end{verbatim}
\end{minipage}
 



